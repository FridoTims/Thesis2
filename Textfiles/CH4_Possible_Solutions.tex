\chapter{Description Investigated Solutions}

\noindent The factors which need further investigation are now compared in a MCA to see which factor is most promising to create a practical solution for. Firstly, a short description of solutions for each factor is presented. Designs that decrease turbidity and that are currently available are shown in Appendix \ref{app:current_designs}.


\newpage
\section{Short description possible solutions}

\textbf{Hopper inlet / sedimentation} \newline
\noindent As said earlier, the best way to decrease surface turbidity theoretically is to let no sediment into the overflow at all. Therefore a solution could be some kind of centrifugal filter setup to filter the sediment and water, where the water can leave the overflow. Instead of the water sediment mixture flows in the overflow, it is sucked up by a pump where it is filtered and the sand can be stored on the TSHD. Therefore no sediment will flow out of the overflow and there will be only a plume caused by the dredge head. Theoretically it is the desirable solution, however, extra room has to be created for the filter setup which can cause practical problems.  \newline

\noindent \textbf{Air / Pulsing} \newline
\noindent Solutions for the intake of air are available which are further explained in appendix \ref{app:current_designs}. However, especially the the green valve has a big disadvantage about the controls during operation which are vulnerable for the sediment in the overflow. Therefore, a simplistic design is preferred which will reduce the pulsing and so the entrapped air into the overflow. \newline A possible solution is (1) a floating overflow, so that the overflow always fills from underwater and no air can be entrapped. In this case, a floating overflow headpiece is placed on a shaft which will increase in height if the hopper is filled. Around the shaft, openings are placed where the water sediment mixture can flow in and due to the rising of the floating overflow headpiece, the openings are only placed underwater so no air can flow into the overflow. This option is however only possible when replacing the whole overflow. Therefore another option is (2) a filter that will stop / reduce air inflow, which is placed on the current overflow inlet piece. \newline

\noindent \textbf{Overflow sediment load} \newline
\noindent Increasing the sediment load in the overflow showed a great decrease of sediment in suspension. This can be achieved by either (1) block the sediment (where water can pass) and release it all together due to a kind of filter layer or (2) inject extra sediment particles to increase the sediment load in the overflow. In this case nozzles are placed in the overflow which will inject extra particles so that the concentration can increase and kept constant. A drawback to this solution is the extra added parts that will need maintenance and are vulnerable to faults during operation.\newline

\noindent \textbf{Overflow shape} \newline
\noindent Different shapes can be tested to see which shape decreases the surface turbidity the most. Additionally, a practical solution to add to current TSHD's overflow should be considered. \newline (1) Change the overflow shape by adding a diffusor that changes shape or (2) mold the best shape in the current overflow system. \newline

\noindent \textbf{Overflow extension} \newline
\noindent As shown, an overflow extension is a good solution to decrease the surface turbidity. However a practical solution is a little bit harder. The extension creates a different flow underwater and must resist the force in the water. Also the connection to the hull, to make the extension a good solution for current TSHD's, should be investigated. \newline Therefore two cases can be considered: (1) an overflow extension that is connected to the hull which can be lifted up by some mechanism. This could be made from either stainless steel, which cannot bend, or a rubberized material so that it could bend with the current to decrease the force on the extension. However, the material should withstand the bending and fatigue which needs further investigation. A second option is (2) an extension which let the sediment flow out at the dredge head. In this case a connection can be made from the overflow outflow to the dredge head so that the sediment leaves the new overflow near bed level. However, in this case, as shown in section \ref{sec:extension}, the re-suspension is so strong that some suspended material reach the keel of the vessel.






%Luchtfilter

%Overflow verlengen (makkelijke bevestiging (klapmechanisme), duurzaam materiaal, niet te hoge kracht (rubberring pa?)  -> Overflow naar zuigbuis

%Iets toevoegen zodat density in overflow verhoogd wordt waardoor materiaal makkelijker en sneller zinkt (zandfilter of iets dergelijks?)

%Shape overflow, verschillende vormen testen

%Siphon idee
%livingstone standpipe idee (zie http://www.hobbykwekers.nl/artikelen/item/6-manieren-van-overlopen)

%Drijvende overflow







%%%%%%%%%%%%%%%%%%%%%%%%%%%%%%%%%%%%%%%%%%%%%%%%%%%%%%%%%%%%%%%%%%%%%%%%%%%%%%%%%%%%%%%%%%%%%%%%%%%%%%%%%%%%%%%%%%%%%%%%%%%%%%%%%%%%%%%%%%%%%%%%%%%%%%%%%%%%

\section{Analysis for further investigation of practical solution}

The proposed solutions are further investigated in a multi criteria analysis. In this analysis, all solutions are compared with each other in a certain amount of factors and each factor is given a weight factor for importance. The MCA is shown in table \ref{tab:MCA}, but first each factor is further elaborated what the meaning is.  \newline

\noindent \textbf{Practicality}\newline
Smaller implementations are more practical than bigger ones. Does the implementation of the solution change the working method of the TSHD or negatively influence the working on a TSHD for the people and the ease of installation of the solution can be noted for this factor. Also a distinction can be made between an active- and passive solution where an active system has moving parts or must be powered where a passive solution doesn't. A good solution can have great efficiency, but also needs practical appliance. Therefore the practicality has a weight factor of 0.25.\newline

\noindent \textbf{Economics} \newline
Another important factor is the total cost of the solution. What will be the cost for the production of the solution and what is the expected lifetime of the product before it has to be replaced are questions that can be asked for this factor. The economics are estimated based on complexity, material and size. To keep a good view at the budget needed for a solution, the economics has a weight factor of 0.2.\newline

\noindent \textbf{Theoretical efficiency}\newline
The theoretical efficiency is obviously relevant for a solution to decrease the surface turbidity. Here, all solutions are elaborated purely based on theoretical efficiency, which was investigated in chapter \ref{CH:influence_factors}. If it is not clear exactly how much the surface turbidity / flux in suspension is decreased, an estimate is based on information of chapter \ref{CH:influence_factors}. As the main drive is to decrease the surface turbidity / sediment in suspension, the theoretical efficiency is weighted highest with a weight factor of 0.3.\newline

\noindent \textbf{Maintainability}\newline
The ease to maintain all solution is clarified here. Some solutions are bigger and have moving parts which need to maintained more often. The ease to reach to the parts which need to be maintained could also be noted. As stated before, the maintainability is important and therefore has a weight factor of 0.15. \newline

\noindent \textbf{Adjustment to TSHD}\newline
Is the solution simple implemented on a TSHD, does the TSHD need some adjustments before implementation, or is it a solution for newly build vessels. As this is not clearly stated in the thesis proposal, this factor is weighted lowest with a factor 0.1. \newline

\noindent \textbf{Test possibility} \newline
How easy can the possibilities be tested in the experimental setups present at APT Offshore or TU Delft. \newline

\noindent All solutions are ranked from 1 to 5, where 5 is ranked as best for that factor and 1 for the worst. As there are 9 solutions, factors can have the same ranked value. 

%\caption{Multi Criteria Analysis with weight factors to determine most promising solution}
%\label{tab:MCA}
%\begin{adjustbox}{max width=\textwidth}
%\end{adjustbox}


\begin{table}[ht]
\centering
\caption{Multi Criteria Analysis with weight factors to determine most promising solution}
\label{tab:MCA}
\begin{adjustbox}{max width=\textwidth}
\begin{tabular}{lcccccccc}
\cline{1-7} \cline{9-9}
\multicolumn{1}{|c|}{\textbf{Solution}}       & \multicolumn{1}{c|}{\textbf{Practicality}} & \multicolumn{1}{c|}{\textbf{Economics}} & \multicolumn{1}{c|}{\textbf{Theoretical efficiency}} & \multicolumn{1}{c|}{\textbf{Maintainability}} & \multicolumn{1}{c|}{\textbf{Adjusment to TSHD}} & \multicolumn{1}{c|}{\textbf{Test possibility}} & \multicolumn{1}{c|}{\textbf{}} & \multicolumn{1}{c|}{\textbf{Total}} \\ \cline{1-7} \cline{9-9} 
\multicolumn{1}{|l|}{Sediment filter on TSHD} & \multicolumn{1}{c|}{3}                     & \multicolumn{1}{c|}{1}                  & \multicolumn{1}{c|}{5}                               & \multicolumn{1}{c|}{2}                        & \multicolumn{1}{c|}{2}                          & \multicolumn{1}{c|}{2}                         & \multicolumn{1}{c|}{}          & \multicolumn{1}{c|}{2,85}           \\ \cline{1-7} \cline{9-9} 
\multicolumn{1}{|l|}{Floating overflow}       & \multicolumn{1}{c|}{3}                     & \multicolumn{1}{c|}{2}                  & \multicolumn{1}{c|}{3}                               & \multicolumn{1}{c|}{3}                        & \multicolumn{1}{c|}{1}                          & \multicolumn{1}{c|}{1}                         & \multicolumn{1}{c|}{}          & \multicolumn{1}{c|}{2,35}           \\ \cline{1-7} \cline{9-9} 
\multicolumn{1}{|l|}{Air filter insert}       & \multicolumn{1}{c|}{5}                     & \multicolumn{1}{c|}{4}                  & \multicolumn{1}{c|}{4}                               & \multicolumn{1}{c|}{4}                        & \multicolumn{1}{c|}{5}                          & \multicolumn{1}{c|}{1}                         & \multicolumn{1}{c|}{}          & \multicolumn{1}{c|}{3,90}           \\ \cline{1-7} \cline{9-9} 
\multicolumn{1}{|l|}{Blockage filter}         & \multicolumn{1}{c|}{2}                     & \multicolumn{1}{c|}{4}                  & \multicolumn{1}{c|}{4}                               & \multicolumn{1}{c|}{4}                        & \multicolumn{1}{c|}{4}                          & \multicolumn{1}{c|}{2}                         & \multicolumn{1}{c|}{}          & \multicolumn{1}{c|}{3,20}           \\ \cline{1-7} \cline{9-9} 
\multicolumn{1}{|l|}{Inject extra sediment}   & \multicolumn{1}{c|}{3}                     & \multicolumn{1}{c|}{3}                  & \multicolumn{1}{c|}{4}                               & \multicolumn{1}{c|}{2}                        & \multicolumn{1}{c|}{3}                          & \multicolumn{1}{c|}{3}                         & \multicolumn{1}{c|}{}          & \multicolumn{1}{c|}{3,15}           \\ \cline{1-7} \cline{9-9} 
\multicolumn{1}{|l|}{Diffusor}                & \multicolumn{1}{c|}{4}                     & \multicolumn{1}{c|}{5}                  & \multicolumn{1}{c|}{3}                               & \multicolumn{1}{c|}{4}                        & \multicolumn{1}{c|}{4}                          & \multicolumn{1}{c|}{5}                         & \multicolumn{1}{c|}{}          & \multicolumn{1}{c|}{\cellcolor{green}4,05}           \\ \cline{1-7} \cline{9-9} 
\multicolumn{1}{|l|}{Shape mold insert}       & \multicolumn{1}{c|}{4}                     & \multicolumn{1}{c|}{5}                  & \multicolumn{1}{c|}{2}                               & \multicolumn{1}{c|}{5}                        & \multicolumn{1}{c|}{5}                          & \multicolumn{1}{c|}{5}                         & \multicolumn{1}{c|}{}          & \multicolumn{1}{c|}{\cellcolor{green}4,00}           \\ \cline{1-7} \cline{9-9} 
\multicolumn{1}{|l|}{Overflow extension}      & \multicolumn{1}{c|}{3}                     & \multicolumn{1}{c|}{3}                  & \multicolumn{1}{c|}{4}                               & \multicolumn{1}{c|}{2}                        & \multicolumn{1}{c|}{3}                          & \multicolumn{1}{c|}{5}                         & \multicolumn{1}{c|}{}          & \multicolumn{1}{c|}{3,45}           \\ \cline{1-7} \cline{9-9} 
\multicolumn{1}{|l|}{Suction pipe overflow}   & \multicolumn{1}{c|}{2}                     & \multicolumn{1}{c|}{1}                  & \multicolumn{1}{c|}{3}                               & \multicolumn{1}{c|}{1}                        & \multicolumn{1}{c|}{2}                          & \multicolumn{1}{c|}{2}                         & \multicolumn{1}{c|}{}          & \multicolumn{1}{c|}{2,00}           \\ \cline{1-7} \cline{9-9} 
\multicolumn{1}{c}{}                          &                                            &                                         &                                                      &                                               &                                                 &                                                &                                &                                     \\ \cline{1-7} \cline{9-9} 
\multicolumn{1}{|c|}{\textit{Weight factor}}           & \multicolumn{1}{c|}{0,25}                  & \multicolumn{1}{c|}{0,15}               & \multicolumn{1}{c|}{0,25}                            & \multicolumn{1}{c|}{0,1}                      & \multicolumn{1}{c|}{0,1}                        & \multicolumn{1}{c|}{0,15}                      & \multicolumn{1}{c|}{}          & \multicolumn{1}{c|}{1,00}           \\ \cline{1-7} \cline{9-9} 
\end{tabular}
\end{adjustbox}
\end{table}






%Elke solution doornemen als conclusie van de MCA

%Maintainabillity onderverdelen in boven/onderwater, onderwater meer werk blabla


%%%%%%%%%%%%%%%%%%%%%%%%%%%%%%%%%%%%%%%%%%%%%%%%%%%%%%%%%%%%%%%%%%%%%%%%%%%%%%%%%%%%%%%%%%%%%%%%%%%%%%%%%%%%%%%%%%%%%%%%%%%%%%%%%%%%%%%%%%%%%%%%%%%%%%%%%%%%
\newpage
\section{Summary}
A total of nine solutions are weighted under certain factors, shown in table \ref{tab:MCA}, where is shown that certain solutions are more preferable than others. Concluding the results of the MCA from weakest to best solution, the \textit{extended overflow to the suction pipe} comes first. Based on the results on economics and maintainability, where it scores the minimum, it results in an overall lowest position. \newline
Next is the \textit{floating overflow}. The floating overflow scores low on the adjustment of the TSHD, because the whole overflow structure has to be replaced to work. This relates also to the low score on the economics. The reducing of air showed a great theoretical efficiency however, the floating overflow scores lower than the air filter insert because it is unknown at this point if a floating overflow will reduce the total efficiency of the filling process in the hopper due to a lower placement of the holes underwater.\newline 
Next is the \textit{Sediment filter}. The sediment filter is theoretical the best solution there is, however, some drawbacks are the cost of the installation, the adjustment that have to be made to the TSHD and the maintainability of the centrifugal filter which adds parts to the current setup of the overflow. \newline
Two solutions which score the same amount are the \textit{Overflow extension} and the \textit{extra injection of sediment} in the overflow. The increase of concentration in the overflow showed good results in \ref{CH:influence_factors} and so a high theoretical efficiency. A drawback is the injection system, which need regular maintenance. The injection system can be added to current TSHD's but the overflow will need adjustments where holes need to be bored for the injection nozzles, which will cause the vessel to maintain at the dock for a time and therefore an average score on economics. In case of the overflow extension, the connection of the extension to the hull of the TSHD is a point of attention, due to the force which will act on the connection. Maintainability scores low because the extension is present beneath water level, so maintenance is harder doable.\newline
The runner ups in this MCA are the \textit{Change of overflow shape} and the \textit{Blockage filter} where the change of overflow shape scores highest on the maintainability, because there is basically no maintenance needed. The adjustment to the TSHD necks the change of overflow shape, because the whole overflow needs to be replaced. Furthermore it scores above-average for the other factors where for new vessels it could be implemented from scratch and therefore scores good for economics and practicality. The blockage filter scores low on the practicality on how to separate sand and water and let the water flow further but scores high on the theoretical efficiency, same as the extra injection of sediment. Theoretical, the blockage filter could be inserted as a passive system in current TSHD's, but the placement and working method complicate the solution. Maintainability scores good, because it is a passive system and therefore also the ease of installation.\newline
Two solutions reach far above all other and so close to each other, that no direct difference can be distinguished between the \textit{Air filter insert} and the \textit{Shape mold insert}. Both solutions score highest on practicality and adjustment to the TSHD because both are passive systems and so can be used of current and new TSHD's. The main difference is, based on the IHC Plumigator® overflow (which is further elaborated in appendix \ref{app:current_designs}), the complexity of the filter and placement. Where the air filter is placed at the top of the overflow, the shape mold insert is placed at the bottom of the overflow. Therefore it has less influence of environmental impact and so scores higher on maintainability. The shape of the air filter is estimated to be more complex and so costs more to produce and therefore is rated lower than the shape mold insert. As shown in chapter \ref{CH:influence_factors}, the theoretical efficiency is higher for the air filter insert but less investigation is done in changing the shape of the overflow exit. Investigation has to be done in the higher velocity in a shape mold insert, due to a smaller cross section and so possible more turbulence creation. \newline

\noindent \textit{Tests in an experimental setup by changing overflow shapes are not done in the PhD thesis of \cite{Decrop} or in the investigation of \cite{Wang}, and so is an interesting part to investigate further.} 


% OVERALL CONCLUSIE EN HOE NU VERDER?
% OVERLEGGEN MET GEORGES EN KEETELS